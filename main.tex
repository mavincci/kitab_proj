% \documentclass[a4paper]{report}
\documentclass{report}
\usepackage[margin=0.75in]{geometry}


\begin{document}

\chapter{Introduction}
\section{Background}

In the digital world, information is being transmitted from anywhere across the world to your hand. Based on this fast-growing transmission of information, society has become more dependent on that information and having it at any time. And it is likely that anyone being online at any time at any location. Not only this, in today’s world digital devices are becoming multi-functional with relatively low cost, small size and easy to use with an attractive user interface. This makes many people use their device for reading soft copy books instead of hard copy.
	
The internet has become a major resource in modern business, thus electronic shopping has gained a significant advantage not only from the entrepreneur’s view but also from the customer’s point of view. Web services are one such area where developers must lean on their creative side and hope that their solutions are still successful. In this project, we will explain an exciting system available in the town of Addis Ababa.

From requirements to use cases, to database design, to component frameworks, to user interfaces, we will cover every aspect of system design required to build an application with collaborative online services. The reason why we selected the online eBook selling system (Kitab) is everybody walking down the street has some idea about bookstores and many people have smart electronic devices and internet access.

So online eBooks selling system is widely used in the outside world. It has provided an environment where everyone can buy and sell books at any time and place. but the system implemented cannot fully address all areas in the world. Due to this the online eBook selling system in the outside world cannot be fully functional in our country due to different problems. That is why we are developing a system which is focused on local market.

This project deals with developing an e-commerce website for eBooks. It provides the user with a catalog of different eBooks available for purchase in the store or gets all eBook information without going through a bookstore. To facilitate online purchase a shopping cart is provided to the user(reader).

This project is a website that acts as a central eBook store for selling, buying, reading, etc.… for readers, authors, publishers or any other stakeholders. When we start developing this project, we are considering all stakeholders including publishers because it has an impact on the physical printing business.  We conducted an interview with publishing company owners, authors, book distributors, and book shop owners. We also prepared a questionnaire and we collected many comments from readers, students, publishing company employees and teachers.

\section{Statement of the problem}
	\subsection{Existing system}

The currently available book store system in Ethiopia is the traditional way (physical bookstores). When a publisher offers the author a contract and the author agree with the contract, in turn, prints, publishes, distributes and sells the book through booksellers and other retailers.

The publisher essentially buys the right to publish the author's book and pays them their share as they agreed in the contract or pays the cost of the book. Most writers, when they want to publish a book, they need to find an agent. To find the agent, they need to prepare their book proposal with some sample chapters. Once they completed the above steps, the need to write a query letter which is for submitting to an agent. The publishers use some printing technology to produce books through a printing company at a cost-effective price.

The books are printed individually as the order comes in to adjust the book's supply to meet the reader's demand.  There is also another method of publishing books in which the publisher publishes anyone’s work provided who have money to pay for the services. The publisher prints and bind the book on the author's money and the authors own the printed books and retain all profit from sales.

There is also a self-publishing mechanism that requires the authors to invest their own money to produce, market promotion and distribute their work. generally, the self-publishing option is one in which the author eliminates some of the middlemen and manages the overall publishing, distribution and marketing processes him/herself. This option gives the author much more personal control of the whole process and allows him/her to earn more money per copy than through a commercial publisher.

	\subsection{Major problem of existing system}
		\subsubsection{In terms of authors}

\begin{itemize}
	\item The authors give away their book’s value in percent. If their book's cover price is, say 300 Birr, they will be forced to discount some percent for publishing and other service they required from other agents. So, most authors did not get enough payment compared to their effort for writing their book.
	\item They don't count on making big profit. If they choose the commercial publisher option, the best they can hope to receive for their book is the net income generated from the book sells in percent, the amount the publisher receives after discounting to retailers.
	\item Waiting for a long period of time to get paid. sometimes, they will have to wait many days or months after an actual book sale before they will receive their payment for that sale. Because the process of checking if all copies of the books are sold or not takes a long time.
	\item Many authors did not get feedbacks about their work from readers. Even if there is an option for the readers to comment, like email, post mail, telephone etc. most authors will not get comments from their audience.
\end{itemize}

		\subsubsection{In terms of publishers}

\begin{itemize}
	\item They need a high investment. Since physically printing books require much technology born materials like paper, printing machine, colors, etc., they need to have lots of money to buy all those materials. And also, publishers need more manpower (human resource) for their publishing tasks.
	\item Inadequate distribution of books through bookshop. The availability of their books on the market is limited. Physical books are available in some specific locations especially large towns. Due to this, their book may not reach many readers across the country.
	\item They may not know if the printed books are sold out. Because the books are sold in many parts of the country with many local sellers, they will have not enough information for printing the new one.
	\item Their books may easily be scanned and converted to softcopies and easily shared. The copy right control system is difficult.
	\item Papers are manufactured from tree. With irresponsible manufacturers it is cause for deforestation.
\end{itemize}

		\subsubsection{In terms of readers}

\begin{itemize}
	\item Readers may have many books. Since the books are paper, they occupy large physical space for storing. So they need more storage space or shelf in their home.
	\item Portability issue. They can’t access their books Everywhere unless they carry it. Books are available in specific locations. If readers want their book available everywhere in their hands, the need to carry it wherever they go. That is not a comfortable way and sometimes impossible.
	\item Not easy on the Eyes. The paper may not be comfortable for outdoor use.
	\item It takes time to get their books of interest. Since paper books are available only in the book store (book shops) customers may not get stores nearby.
	\item The books are relatively available at high prices. Paper books are often expensive in the long run because there are printing fees and the material associated with them.
	\item It is not possible to read at night unless using artificial lights which additional cost.
	\item They may not get information about new books in a short time.
	\item Readers cannot search for items in the book easily. They need to browse through the book's papers, maybe page by page to get a specific item in the book. So, searching for pages and contents on a paper book is laborious work.
	\item Lack of adequate information about the bookshop.
\end{itemize}

		\subsection{Proposed system}

We have proposed a system that will solve the above problems mentioned. Let’s look at some of the advantages of Kitab from a different perspective as follows

\begin{itemize}
	\item{\Large{NONE TODO}}
\end{itemize}	

		\subsection{Advantage of proposed system}

\begin{itemize}
	\item It allows authors to make effective marketing by selling their book directly to their readers without need to get a publishing agent.
	\item Allows publishers to sell their books in eBook format in addition to their hard copy.
	\item Allow readers to easily get their book of interest in eBook format without the need to go to the books store.
	\item It allows greater control over sales.
	\item Easy to travel with books. allow readers to bring a whole library with them wherever they go.
	\item It saves physical shelves storage from being full of many papers.
	\item It saves time for authors, readers, publishers by decreasing sales, publishing and browsing time.
	\item It decreases the price of books.
	\item It is possible to read at night since it is on electronic devices.
	\item It is accessible everywhere 
	\item When the device lost, it is possible to get the eBooks back. So, it allows lost books easily recoverable
	\item Readers can easily search for any information in an eBook, instead of turning page after page.
\end{itemize}

	\section{Motivation}

Recently, the internet has become a great emerging business sector across the world. online learning, electronic marketplace, online store, online information providers and enterprises of the electronic information exchange … and so on. We see many kinds of e-commerce web shopping sites selling various goods. Internet access in Ethiopia has increased in both speed and coverage. But for many years there is only traditional book publishing technology available. It needs more cost for printing and binding books, takes more time to complete publication and distribution, inaccessible everywhere, price for one is high, causes the authors economically un advantageous and many more problems. Our main motivation was to solve this much trouble. Seeing this cost, we thought of a system that could eliminate such costs thereby providing other related features. The existence of a local online bookselling system for Ethiopian readers, authors and publishers is the best solution for all this problem. We have also realized that even if there are many people who want to read books, they can get it due to books price, unavailability in the local area. Thus, we thought of a system with a central data storage where books are stored in an eBook format and anyone can buy. And enabling authors or publishers to sell the eBook form of their book online.

	\section{Scope and limitation of the project}
		\subsection{Scope of the project}

The system that we are going to designed and implemented, enables readers to have a book shop in their hands. And to have their books stored on their device local storage and in the cloud, which is a safe and central place. This project initially focuses on readers who have devices with internet access which have the ability to read eBooks, authors who want to sell their book directly to consumers and publishers who want the online market. In the near future, we aim to extend the system so that it will accommodate each and every book consumers and sellers in the whole of Ethiopia.

This project work is narrowed to the online book shop and central repository system of most locally written books. The names of the available books and their prices are listed.

Our project allows readers to better understand how online shopping works locally and thus empower them to read more. and prevent them from being tired of finding book shops and ratings at a high cost. Our project will also allow authors and publishers to sell and buy books confidentially.

It allows all users to comment and give feedback about books and about the system. They can get quick help with the system. 

It will have a categorical book shelving system for easy browsing, enable reviewing samples before buying, offer suggestions based on the most read and liked books.

To achieve our general and specific objectives we have categorized our project into three main stages:

\begin{enumerate}
	\item The first stage associated with requirement identification and analysis. We will collect most of the data used for our project from the publishers, authors, readers and students.
	
	\item The second stage of the project is system design which consists of:	
	\begin{itemize}
		\item Identify design goals
		\item Hardware – software mapping
		\item Persistent data management
		\item Build and architectural model for the system
	\end{itemize}

	\item The third stage will focus on the actual system implementation and testing. This includes:
	\begin{itemize}
		\item Implement the system using the selective software tools on the appropriate hardware platform
		\item Carry out unit and integration tests
		\item Prepare user manual
		\item Make it ready to use
	\end{itemize}
\end{enumerate}

	\subsection{Limitation of the project}

\begin{itemize}
\item The system does not serve for buying hard copy books like other e-commerce sites have.
\item The system does not support all local languages and other foreign language other than English.
\end{itemize}

	\section{Objective}
		\subsection{General objective}

Our goal is to build a robust and effective system for Ethiopian online eBook market.

		\subsection{Specific objectives}

\begin{itemize}

	\item Understand the feeling of authors, readers and publishers about using the system to be developed
	\item Understand the current trend of how books are published and distributed
	\item Collect and analyze data from users’ respective 
	\item Investigate whether software system is applicable to solve the problem considering political, economic and social issues,
	\item Collect requirements from the users and analyze them.
	\item Design a system that will solve the problems that are found in current available system and implement it.
	\item Test the developed system against the collected requirements.
	\item Deploy the system and make it available to users
	\item Increase the awareness and understanding of the online book shop system in Ethiopia.
	\item Achieve efficient and reasonable regulation of online book shop
	\item To create a room where customers can view adequate information about books and buy books and sellers can sell their work to their customers.
	\item Expand the global acceptance of the online book shop
	\item Assist customers in buying books online

\end{itemize}

	\section{Methodologies and technologies}
		\subsection{Data collection}

It's impossible to develop a software system without clear understanding of the problem domain and the requirements of users of the system. As result we elicit requirements to get a good understanding of the problem in hand.

Since our project is on online book store and selling, our main area of data collection revolves around discovering how Ethiopian books are being produced and distributed to readers in the country. Having this in mind we used some information gathering techniques. The first one is observation. We distributed questionnaires to students, authors and other parts of the society and collected their response to get a good insight of their reading experience, how they access books and even their feeling if the system we are proposing. Consequently, we interviewed authors and publishers about the current system of publishing and distributing books, pros and cons of it, if there is a digital system or they have ever tried to digitalize their work and their feelings if we offer the system. Since we are members of the society, as a reader, we used some experience about the system.

		\subsection{System analysis and design tools}

		\subsection{System development tools}


\end{document}}