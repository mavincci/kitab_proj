\chapter{Introduction}
\section{Background}
Nowadays information is becoming the main resource in every day activities.Fortunately it's easy to access information available for users from wherever it's every where in the world.The Internet has became a major resource in modern business.There are a of users searching the Internet every day and the number of users is increasing day to day.There are 4.1 billion Internet users in the world as of December 2018. This is compared to 3.9 billion Internet users in mid-2018 and about 3.7 billion Internet users in late 2017 [The first link].There are a lot of reasons why users browse the Internet, among others for chatting with any friend,to download books, movies,music, software for searching specific information.


Technology and mobility have influenced every step of consumer’s life, including the way they read books. Book readers have started to change their reading habits, opting for different types of formats of books, such as e-books. An e-book, also known as an electronic or digital book, is a digitally released version of a book, often consisting of text and images and available on electronic devices, such as specifically designed e-book readers [The second link]. In today’s world digital devices are becoming multi-functional with relatively low cost, small size and easy to use with an attractive user interfaces. This makes it easy for users  to read  soft copy books.
Online eBooks selling system is being  used in the different countries of the  world. It has provided an environment where everyone can buy and sell books at any time and place, but the system implemented is not applicable in all areas of the world.Our country is the main victim of this problem.There is no any system for purchasing or reading books here in Ethiopia.That is why we are developing a system which is focused on local market.


\section{Statement of the problem}
The following information is obtained from interviewing Ethiopian Authors association manager(Yezina Worku) and Mr. Ashenafi Taye 
	\subsection{Existing system}

The currently available book store system in Ethiopia is the traditional way (physical bookstores). When a publisher offers the author a contract and the author agree with the contract, in turn, prints, publishes, distributes and sells the book through booksellers and other retailers.

The publisher essentially buys the right to publish the author's book and pays them their share as they agreed in the contract or pays the cost of the book. Most writers, when they want to publish a book, they need to find an agent. To find the agent, they need to prepare their book proposal with some sample chapters. Once they completed the above steps, the need to write a query letter which is for submitting to an agent. The publishers use some printing technology to produce books through a printing company at a cost-effective price.

The books are printed individually as the order comes in to adjust the book's supply to meet the reader's demand.  There is also another method of publishing books in which the publisher publishes anyone’s work provided who have money to pay for the services. The publisher prints and bind the book on the author's money and the authors own the printed books and retain all profit from sales.

There is also a self-publishing mechanism that requires the authors to invest their own money to produce, market promotion and distribute their work. generally, the self-publishing option is one in which the author eliminates some of the middlemen and manages the overall publishing, distribution and marketing processes him/herself. This option gives the author much more personal control of the whole process and allows him/her to earn more money per copy than through a commercial publisher.

	\subsection{Major problem of existing system}
		\subsubsection{From authors perspective}

\begin{itemize}
	\item The authors give away their book’s value in percent. If their book's cover price is, say 300 Birr, they will be forced to discount some percent for publishing and other service they required from other agents. So, most authors did not get enough payment compared to their effort for writing their book.
	\item They don't count on making big profit. If they choose the commercial publisher option, the best they can hope to receive for their book is the net income generated from the book sells in percent, the amount the publisher receives after discounting to retailers.
	\item Waiting for a long period of time to get paid. sometimes, they will have to wait many days or months after an actual book sale before they will receive their payment for that sale. Because the process of checking if all copies of the books are sold or not takes a long time.
	\item Many authors did not get feedbacks about their work from readers. Even if there is an option for the readers to comment, like email, post mail, telephone etc. most authors will not get comments from their audience.
\end{itemize}

		\subsubsection{From publishers perspective}

\begin{itemize}
	\item They need a high investment. Since physically printing books require much technology born materials like paper, printing machine, colors, etc., they need to have lots of money to buy all those materials. And also, publishers need more manpower (human resource) for their publishing tasks.
	\item Inadequate distribution of books through bookshop. The availability of their books on the market is limited. Physical books are available in some specific locations especially large towns. Due to this, their book may not reach many readers across the country.
	\item They may not know if the printed books are sold out. Because the books are sold in many parts of the country with many local sellers, they will have not enough information for printing the new one.
	\item Their books may easily be scanned and converted to soft copies and easily shared. The copy right control system is difficult.
	\item Papers are manufactured from tree. With irresponsible manufacturers it is cause for deforestation.
\end{itemize}

		\subsubsection{From readers perspective}

\begin{itemize}
	\item Readers may have many books. Since the books are paper, they occupy large physical space for storing. So they need more storage space or shelf in their home.
	\item Portability issue. They can’t access their books Everywhere unless they carry it. Books are available in specific locations. If readers want their book available everywhere in their hands, the need to carry it wherever they go. That is not a comfortable way and sometimes impossible.
	\item Not easy on the Eyes. The paper may not be comfortable for outdoor use.
	\item It takes time to get their books of interest. Since paper books are available only in the book store (book shops) customers may not get stores nearby.
	\item The books are relatively available at high prices. Paper books are often expensive in the long run because there are printing fees and the material associated with them.
	\item It is not possible to read at night unless using artificial lights which additional cost.
	\item They may not get information about new books in a short time.
	\item Readers cannot search for items in the book easily. They need to browse through the book's papers, maybe page by page to get a specific item in the book. So, searching for pages and contents on a paper book is laborious work.
	\item Lack of adequate information about the bookshop.
\end{itemize}

		\subsection{Proposed system}

We have proposed a system that will minimize the problems mentioned above. This project we are developing  a system(Kitab) that acts as a central eBook store for selling, buying, reading, etc.… for readers, authors and publishers.Kitab enable users to read E-books using their smart phone. It provides the user with a catalog of different eBooks available for purchase in the store or gets all eBook information without going through a bookstore.Additionally it will enable users to search for E-books,to browse through category of E-books .To facilitate online purchase a shopping cart is provided to the user(reader).This system will increase the number of readers by enabling them to access books which they may not get by using the current system,since the system will be accessible every where which thereby enable authors and publishers get more profit.

Let’s look at some of the advantages of Kitab from a different perspective as follows

% \begin{itemize}
% 	\item{\Large{NONE TODO}}
% \end{itemize}	

		\subsection{Advantage of proposed system}

\begin{itemize}
	\item It allows authors to make effective marketing by selling their book directly to their readers without need to get a publishing agent.
	\item Allows publishers to sell their books in eBook format in addition to their hard copy.
	\item Allow readers to easily get their book of interest in eBook format without the need to go to the books store.
	\item It allows greater control over sales.
	\item Easy to travel with books. allow readers to bring a whole library with them wherever they go.
	\item It saves physical shelves storage from being full of many papers.
	\item It saves time for authors, readers, publishers by decreasing sales, publishing and browsing time.
	\item It decreases the price of books.
	\item It is possible to read at night since it is on electronic devices.
	\item It is accessible everywhere 
	\item When the device lost, it is possible to get the eBooks back. So, it allows lost books easily recoverable
	\item Readers can easily search for any information in an eBook, instead of turning page after page.
\end{itemize}

	\section{Motivation}

Recently, the internet has become a great emerging business sector across the world. online learning, electronic marketplace, online store, online information providers and enterprises of the electronic information exchange … and so on. We see many kinds of e-commerce web shopping sites selling various goods. Internet access in Ethiopia has increased in both speed and coverage. But for many years there is only traditional book publishing technology available. It needs more cost for printing and binding books, takes more time to complete publication and distribution, inaccessible everywhere, price for one is high, causes the authors economically un advantageous and many more problems. Our main motivation was to solve this much trouble. Seeing this cost, we thought of a system that could eliminate such costs thereby providing other related features. The existence of a local online bookselling system for Ethiopian readers, authors and publishers is the best solution for all this problem. We have also realized that even if there are many people who want to read books, they can get it due to books price, unavailability in the local area. Thus, we thought of a system with a central data storage where books are stored in an eBook format and anyone can buy. And enabling authors or publishers to sell the eBook form of their book online.

	\section{Scope and limitation of the project}
		\subsection{Scope of the project}

The system we are designing enables peoples to have a book shop in their hands.It stores books on their devices' local storage and in the cloud, which is a safe and central place. This project primarily focuses on peoples who have smart phones with internet access and have the ability to read eBooks, authors who want to sell their book directly to readers and publishers who want the online market. This project work is narrowed to the online book shop and central repository system. In the near future, we aim to extend the system so that it will accommodate each and every book consumers and sellers in the whole of Ethiopia. Our project allows readers to better understand how online shopping works locally and thus empower them to read more. Additionally prevent them from being tired of finding book shops and ratings at a high cost.

It allows all users to rate and give feedback about the content they visited or bought. This will help authors and publishers sell their books with confidence. Readers can get enough information about the content by reading the reviews. It will have a categorical book shelving system for easy browsing, enable reviewing samples before buying, offer suggestions based on the most read and liked books.

To achieve our general and specific objectives we have categorized our project into three main stages:

\begin{enumerate}
	\item The first stage associated with requirement identification and analysis. We will collect most of the data used for our project from the publishers, authors, readers and students.
	
	\item The second stage of the project is system design which consists of:	
	\begin{itemize}
		\item Identify design goals
		\item Hardware – software mapping
		\item Persistent data management
		\item Build and architectural model for the system
	\end{itemize}

	\item The third stage will focus on the actual system implementation and testing. This includes:
	\begin{itemize}
		\item Implement the system using the selective software tools on the appropriate hardware platform
		\item Carry out unit and integration tests
		\item Prepare user manual
		\item Make it ready to use
	\end{itemize}
\end{enumerate}

	\subsection{Limitation of the project}

\begin{itemize}
\item The system does not serve for buying hard copy books like other e-commerce sites have.
\item The system does not support all local languages and other foreign language other than English.
\end{itemize}

	\section{Objective}
		\subsection{General objective}

Our goal is to build a robust and effective system for Ethiopian online eBook market.

		\subsection{Specific objectives}

\begin{itemize}

	\item Understand the feeling of authors, readers and publishers about using the system to be developed
	\item Understand the current trend of how books are published and distributed
	\item Collect and analyze data from users’ respective 
	\item Investigate whether software system is applicable to solve the problem considering political, economic and social issues,
	\item Collect requirements from the users and analyze them.
	\item Design a system that will solve the problems that are found in current available system and implement it.
	\item Test the developed system against the collected requirements.
	\item Deploy the system and make it available to users
	\item Increase the awareness and understanding of the online book shop system in Ethiopia.
	\item Achieve efficient and reasonable regulation of online book shop
	\item To create a room where customers can view adequate information about books and buy books and sellers can sell their work to their customers.
	\item Expand the global acceptance of the online book shop
	\item Assist customers in buying books online

\end{itemize}

	\section{Methodologies and technologies}
		\subsection{Data collection}

It's impossible to develop a software system without clear understanding of the problem domain and the requirements of users of the system. As result we elicit requirements to get a good understanding of the problem in hand.

Since our project is on online book store and selling, our main area of data collection revolves around discovering how Ethiopian books are being produced and distributed to readers in the country. Having this in mind we used some information gathering techniques. The first one is observation. We distributed questionnaires to students, authors and other parts of the society and collected their response to get a good insight of their reading experience, how they access books and even their feeling if the system we are proposing. Consequently, we interviewed authors and publishers about the current system of publishing and distributing books, pros and cons of it, if there is a digital system or they have ever tried to digitalize their work and their feelings if we offer the system. Since we are members of the society, as a reader, we used some experience about the system.

		\subsection{System analysis and design tools}

For this project we used 

\begin{description}
	\item[MikTex] - a \LaTeX distribution for windows. We used it prepare our document.
	\item[Enterprise architect] - a professional UML modelling tool. We used it to prepare our design diagrams.
	\item[Visual paradigm] - also a UML modelling tool. We used it to prepare some of our design diagrams.
	\item[Gantt project] - to prepare our schedule by generating a gantt chart.
\end{description}

		\subsection{System development tools}

For the implementation part we are planning to use the tools listed below.
\begin{itemize}
	\item Development tools
	\begin{description}
		\item[Visual-Studio code] - very popular free and opensource source code editor with seamless integration with go and javascript.
		\item[Google chrome] - also very popular browser. We use it to test our web frontend.
		\item[Android studio] - a robust integrated development enviroment(IDE) that is purposely built for developing android applications. It is built on top of the Jetbrains platform. We use it to build our mobile client.
		\item[Git] - an free and opensource distributed version controll system
	\end{description}
	\item Development plaforms and frameworks
	\begin{description}
		\item[golang with Gorilla and Gin-gonic] - to develop our all the backend logic.
		\item[Javascript with ReactJS] - to develop our web front end for the browser.
		\item[Dart with Flutter] - new and mature framework to develop cross-platform applications. Developed by google. We use it to develope our mobile client application.
		\item[Postgresql] - free and open-source database management system. We use it as a backend to all the services.
		\item[Sqlite] - to store mobile client data on the mobile phone.
	\end{description}


\end{itemize}
