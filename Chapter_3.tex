\chapter{System Analysis and Modelling}
\section{Overview}
In this chapter, we are going to use different modelling and diagramming techniques to better understand and describe
the gathered requirements for kitab. For the same purpose, we considered the modelling techniques listed below.

	\begin{description}
		\item[Scenario based modelling] - with UML is the technique begins with the creation of different scenarios in order to understand the problem to be solved. Done by using use-case diagram and activity diagram.
		\item[Behavioural/Dynamic modelling] - by using Sequence diagram and state diagram.
		\item[Class-based modelling] - by using class diagram.
	\end{description}

\section{Scenario Based Modelling}

In this section we will determine and model specific procedures of the system, who take part in these procedures and steps to achieve goals of each procedure.

	\subsection{Use-case identification}

	\begin{itemize}
		\item Add Author
		\item Add Comment
		\item Add Publisher
		\item Authenticate User
		\item Check Payment
		\item Create Account
		\item Download content
		\item Get Help
		\item Modify Content
		\item Rate Content
		\item Recover Password
		\item Remove Account
		\item Remove Content
		\item Search Content
		\item Update Accoutnt-details
		\item Update Content
		\item View Account-Detail
		\item View Content-Review
		\item View System-Statistics
	\end{itemize}

	\subsection{Actor identification}

	\begin{description}
		\item[Admin] - This actor is responsible for managing the system at a higher privilege. Admin can manage content and accounts.
		\item[Publisher] - Publisher is responsible to upload and remove contents and also give recognition to authors. It is added by the admin.
		\item[Reader] - Reader is the main user of the system it can download and read contents and also buy them on local devices. And also give rating and review of a content.
		\item[Author] - Author prepares a content and upload them to the system. It can browse the reviews and rating.
	\end{description}

	\subsection{Use-case diagrams}

\begin{center}
	\includegraphics[width=10cm]{{"Diagram/Manage Account"}.png}

	\includegraphics[width=10cm]{{"Diagram/Manage Content"}.png}
	
	\includegraphics[width=15cm]{{"Diagram/Manage User"}.png}
	
	\includegraphics[width=15cm]{Diagram/Market.png}
	
	\includegraphics[width=15cm]{Diagram/Report.png}
\end{center}


	\subsection{Use-case description}

% \begin{table}
% \begin{center}
\begin{tabular}{| l | c |}
	\hline \textbf{Use-case name} & xyz \\
	\hline \textbf{Actor} & xyz \\
	\hline \textbf{Used use-case} & xyz \\
	\hline \textbf{Goal of context} & xyz \\
	\hline \textbf{Pre-condition} & xyz \\
	\hline \textbf{Post-condition} & xyz \\
	\hline \textbf{Exception} & xyz \\
	\hline \textbf{Flow of events} & xyz \\
	\hline
\end{tabular}
% \end{center}
% \end{table}

	\subsection{Activity diagram}

	
An activity diagram describes the control flow from a start point to a finish point showing the various decision paths that exist while the activity is being executed.The activity diagram for Kitab is categorized based on the Actor.


\begin{figure}[H]
\begin{center}	

	\tcbox{\includegraphics[width=0.4\textwidth]{Diagram/activity/Add Publisher.png}}
	\caption{Activity diagram for adding publisher.}
	\label{dia_actvt_addpblshr}

\end{center}
\end{figure}

\begin{figure}[H]
\begin{center}	

	\tcbox{\includegraphics[width=0.6\textwidth]{Diagram/activity/Login.png}}
	\caption{Activity diagram for login.}
	\label{dia_actvt_login}

\end{center}
\end{figure}

\begin{figure}[H]
\begin{center}	

	\tcbox{\includegraphics[width=0.6\textwidth]{Diagram/activity/Register.png}}
	\caption{Activity diagram for registering.}
	\label{dia_actvt_rgstr}

\end{center}
\end{figure}

\begin{figure}[H]
\begin{center}	

	\tcbox{\includegraphics[width=0.6\textwidth]{Diagram/activity/Remove Account.png}}
	\caption{Activity diagram for removing account.}
	\label{dia_actvt_rmvaccnt}

\end{center}
\end{figure}

\begin{figure}[H]
\begin{center}	

	\tcbox{\includegraphics[width=0.2\textwidth]{Diagram/activity/Remove Content.png}}
	\caption{Activity diagram for removing content.}
	\label{dia_actvt_rmvcntnt}

\end{center}
\end{figure}

\begin{figure}[H]
\begin{center}	

	\tcbox{\includegraphics[width=0.6\textwidth]{Diagram/activity/Review.png}}
	\caption{Activity diagram for reviewing.}
	\label{dia_actvt_rvw}

\end{center}
\end{figure}

\begin{figure}[H]
\begin{center}	

	\tcbox{\includegraphics[width=0.4\textwidth]{Diagram/activity/Search Account.png}}
	\caption{Activity diagram for searching account.}
	\label{dia_actvt_srchaccnt}

\end{center}
\end{figure}

\begin{figure}[H]
\begin{center}	

	\tcbox{\includegraphics[width=0.4\textwidth]{Diagram/activity/Search Content.png}}
	\caption{Activity diagram for searching content.}
	\label{dia_actvt_srchcntnt}

\end{center}
\end{figure}

\begin{figure}[H]
\begin{center}	

	\tcbox{\includegraphics[width=0.4\textwidth]{Diagram/activity/Update Account.png}}
	\caption{Activity diagram for updating account.}
	\label{dia_actvt_updtaccnt}

\end{center}
\end{figure}

\begin{figure}[H]
\begin{center}	

	\tcbox{\includegraphics[width=0.6\textwidth]{Diagram/activity/Update Content.png}}
	\caption{Activity diagram for updating content.}
	\label{dia_actvt_updtcntnt}

\end{center}
\end{figure}

\begin{figure}[H]
\begin{center}	

	\tcbox{\includegraphics[width=0.6\textwidth]{Diagram/activity/Upload File.png}}
	\caption{Activity diagram for uploading file.}
	\label{dia_actvt_upldfl}

\end{center}
\end{figure}

\begin{figure}[H]
\begin{center}	

	\tcbox{\includegraphics[width=0.6\textwidth]{Diagram/activity/Verify Password.png}}
	\caption{Activity diagram for verifying password.}
	\label{dia_actvt_vrfypsswd}

\end{center}
\end{figure}

\begin{figure}[H]
\begin{center}	

	\tcbox{\includegraphics[width=0.6\textwidth]{Diagram/activity/View Account-details.png}}
	\caption{Activity diagram for viewing account details.}
	\label{dia_actvt_vwaccntdtls}

\end{center}
\end{figure}



































\begin{figure}[H]
\begin{center}	

	\tcbox{\includegraphics[width=\textwidth]{Diagram/activity/For Admin.png}}
	\caption{Activity diagram for admin.}
	\label{dia_actvt_fradmn}

\end{center}
\end{figure}

\begin{figure}[H]
\begin{center}	

	\tcbox{\includegraphics[width=\textwidth]{Diagram/activity/For Publisher.png}}
	\caption{Activity diagram for publisher.}
	\label{dia_actvt_frpblshr}

\end{center}
\end{figure}

\begin{figure}[H]
\begin{center}	

	\tcbox{\includegraphics[width=\textwidth]{Diagram/activity/For Reader.png}}
	\caption{Activity diagram for reader.}
	\label{dia_actvt_frrdr}

\end{center}
\end{figure}



\section{Behavioural/Dynamic Modelling}

Behavioural/Dynamic modelling is a system of modeling the dynamic behavior of a system as it is executing. It shows what happens or what is supposed to happen when a system responds to a stimulus from its environment.

These stimuli may be either data or events:
	\begin{enumerate}
		\item Data becomes available that has to be processed by the system. The availability of the data triggers the processing.
		\item An event happens that triggers system processing. Events may have associated data, although this is not always the case.
	\end{enumerate}

	\subsection{Sequence diagram}

	
Sequence diagrams are Data-driven models, which are used to model interactions between system components, although external agents may also be included. A sequence diagram shows the sequence of interactions that take place during a particular use case or use case instance.

\begin{figure}[H]
\begin{center}	

	\tcbox{\includegraphics[width=15cm]{Diagram/sequence/Create Account.png}}
	\caption{Sequence diagram for creating account.}
	\label{dia_sqns_crtaccnt}

\end{center}
\end{figure}

\begin{figure}[H]
\begin{center}	

	\tcbox{\includegraphics[width=15cm]{Diagram/sequence/Add Content.png}}
	\caption{Sequence diagram for adding content.}
	\label{dia_sqns_addcntnt}

\end{center}
\end{figure}


\begin{figure}[H]
\begin{center}	

	\tcbox{\includegraphics[width=15cm]{Diagram/sequence/Download Content.png}}
	\caption{Sequence diagram for downloading content.}
	\label{dia_sqns_dwnldcntnt}

\end{center}
\end{figure}

\begin{figure}[H]
\begin{center}	

	\tcbox{\includegraphics[width=15cm]{Diagram/sequence/Remove Content.png}}
	\caption{Sequence diagram for removing content.}
	\label{dia_sqns_rmvcntnt}

\end{center}
\end{figure}


	\subsection{State diagram}

	
\begin{figure}[H]
\begin{center}	

	\tcbox{\includegraphics[width=15cm]{Diagram/sequence/Create Account.png}}
	\caption{State diagram for creating account.}
	\label{dia_stt_crtacnt}

\end{center}
\end{figure}

\begin{figure}[H]
\begin{center}	

	\tcbox{\includegraphics[width=15cm]{Diagram/sequence/Add Content.png}}
	\caption{State diagram for adding content.}
	\label{dia_stt_addcntnt}

\end{center}
\end{figure}

\begin{figure}[H]
\begin{center}	

	\tcbox{\includegraphics[width=15cm]{Diagram/sequence/Download Content.png}}
	\caption{State diagram for downloading content.}
	\label{dia_stt_dwnldcntnt}

\end{center}
\end{figure}

\begin{figure}[H]
\begin{center}	

	\tcbox{\includegraphics[width=15cm]{Diagram/sequence/Remove Content.png}}
	\caption{State diagram for removing content.}
	\label{dia_stt_rmvcntnt}

\end{center}
\end{figure}

   \pagebreak
   
\section{Class-Based Modeling}

Structural models of software display the organization of a system in terms of the components that make up that system and their relationships.

\textbf{Class-diagrams} for modeling the static structure of the object classes in a software system are used when developing an object-oriented system model to show the classes in a system and the associations between these classes.

	\subsection{Identifying classes}

   \begin{itemize}
   \item{User}
   \item{Content}
   \item{PaidContent}
   \item{Comment}
   \item{Address}
   \item{Admin}
   \item{Publisher}
   \item{History}
   \end{itemize}
   
   
	\subsection{Class diagram}
	
		\begin{figure}[t]
		\begin{center}
	
		\tcbox{\includegraphics[angle=270, width=0.9\textwidth]{Diagram/Class Diagram.png}}
		\caption{Kitab Class diagram}
		\label{dia_class}
	
		\end{center}
		\end{figure}