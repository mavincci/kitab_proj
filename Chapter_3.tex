\chapter{System Analysis and Modelling}
\section{Overview}

In this chapter, we are going to use different modelling and diagramming techniques to better undertand and describe the gathered requirements for kitab.

We will use modelling techniques listed below

	\begin{description}
		\item[Scenario based modelling] - by using use-case diagram and activity diagram.
		\item[Behavioural/Dynamic modelling] - by using Sequence diagram and state diagram.
		\item[Class-based modelling] - by using class diagram.
	\end{description}

\section{Scenario Based Modelling}

In this section we will determine and model specific procedures of the system, who take part in these procedures and steps to achieve goals of each procedure.

	\subsection{Use-case identification}

	\begin{itemize}
		\item Add Author
		\item Add Comment
		\item Add Publisher
		\item Authenticate User
		\item Check Payment
		\item Create Account
		\item Download content
		\item Get Help
		\item Modify Content
		\item Rate Content
		\item Recover Password
		\item Remove Account
		\item Remove Content
		\item Search Content
		\item Update Accoutnt-details
		\item Update Content
		\item View Account-Detail
		\item View Content-Review
		\item View System-Statistics
	\end{itemize}

	\subsection{Actor identification}

	\begin{description}
		\item[Admin] - by using use-case diagram and activity diagram.
		\item[Publisher] - by using use-case diagram and activity diagram.
		\item[Reader] - by using Sequence diagram and state diagram.
		\item[Author] - by using Sequence diagram and state diagram.
	\end{description}

	\subsection{Use-case diagrams}

\begin{center}
	\includegraphics[width=10cm]{{"Diagram/Manage Account"}.png}

	\includegraphics[width=10cm]{{"Diagram/Manage Content"}.png}
	
	\includegraphics[width=15cm]{{"Diagram/Manage User"}.png}
	
	\includegraphics[width=15cm]{Diagram/Market.png}
	
	\includegraphics[width=15cm]{Diagram/Report.png}
\end{center}


	\subsection{Use-case description}

% \begin{table}
% \begin{center}
\begin{tabular}{| l | c |}
	\hline \textbf{Use-case name} & xyz \\
	\hline \textbf{Actor} & xyz \\
	\hline \textbf{Used use-case} & xyz \\
	\hline \textbf{Goal of context} & xyz \\
	\hline \textbf{Pre-condition} & xyz \\
	\hline \textbf{Post-condition} & xyz \\
	\hline \textbf{Exception} & xyz \\
	\hline \textbf{Flow of events} & xyz \\
	\hline
\end{tabular}
% \end{center}
% \end{table}

	\subsection{Activity diagram}



\section{Behavioural/Dynamic Modelling}
	\subsection{Sequence diagram}
	\subsection{State diagram}

\section{Class-Based Modeling}
	\subsection{Identifying classes}
	\subsection{Class diagram}
