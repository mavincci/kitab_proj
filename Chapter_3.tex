\chapter{System Analysis and Modelling}
\section{Overview}

In this chapter, we are going to use different modelling and diagramming techniques to better undertand and describe the gathered requirements for kitab.

We will use modelling techniques listed below

	\begin{description}
		\item[Scenario based modelling] - by using use-case diagram and activity diagram.
		\item[Behavioural/Dynamic modelling] - by using Sequence diagram and state diagram.
		\item[Class-based modelling] - by using class diagram.
	\end{description}

\section{Scenario Based Modelling}

In this section we will determine and model specific procedures of the system, who take part in these procedures and steps to achieve goals of each procedure.

	\subsection{Use-case identification}

	\begin{itemize}
		\item Create account
		\item Upload content
		\item Give feedback and rating
		\item Download content
		\item Search and filter content
	\end{itemize}

	\subsection{Actor identification}

	\begin{description}
		\item[Publisher] - by using use-case diagram and activity diagram.
		\item[User] - by using Sequence diagram and state diagram.
		\item[Author] - by using Sequence diagram and state diagram.
	\end{description}

	\subsection{Use-case description}
	\subsection{Activity diagram}

\section{Behavioural/Dynamic Modelling}
	\subsection{Sequence diagram}
	\subsection{State diagram}

\section{Class-Based Modeling}
	\subsection{Identifying classes}
	\subsection{Class diagram}
